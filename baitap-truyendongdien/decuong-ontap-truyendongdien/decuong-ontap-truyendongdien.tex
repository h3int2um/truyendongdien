\documentclass[12pt,a4paper]{article}
\usepackage[utf8]{inputenc}
\usepackage[utf8]{vietnam}
\usepackage{amsmath,amsfonts,amssymb}
\usepackage{type1cm}
\usepackage{graphicx}
\graphicspath{ {images/} }
\usepackage[unicode]{hyperref}
\usepackage{indentfirst}
\usepackage{color}
\usepackage[left=2.5cm,right=2.5cm,top=2cm,bottom=2cm]{geometry}
\usepackage{enumerate}
\usepackage{hyperref}
\begin{document}
\title{\textbf{\textit{Ôn tập Truyền động điện}}}
\author{GVHD: Hồ Minh Nhị \and SVTH: Thi Minh Nhựt}
\date{\textit{Thời gian:} \today}
\maketitle
\begin{enumerate}
	\item Moment động cơ là đại lượng gì?
	
	\item Moment quán tính, tính toán/quy đổi tương đương trong hệ truyền động điện?
	
	\item Phương trình cân bằng trong chuyển động quay?
	
	\item Đường đặc tính cơ và điểm làm việc ổn định?
	
	\item Chế độ máy phát của động cơ DC, động cơ AC? Cho ví dụ.
	
	\item Biểu thức tính moment động cơ DC và động cơ AC?
	
	\item Mối quan hệ giữa moment, công suất và tốc độ?
	
	\item Đảo chiều động cơ DC kích từ độc lập và nối tiếp?
	
	\item Thay đổi tốc độ bằng bộ chopper, biếu thức tính toán điện áp đặt vào phần ứng, phạm vi điều chỉnh độ Duty cycle?
	
	\item Thay đổi tốc độ bằng bộ chỉnh lưu điều khiển, biếu thức tính toán điện áp đặt vào phần ứng, phạm vi điều chỉnh góc kích?
	
	\item Tại sao động cơ DC kích từ nối tiếp không được vận hành non tải?
	
	\item Các cách sử dụng động cơ 3 pha từ nguồn 1 pha, vẽ mạch nguyên lý?
	
	\item Hãm động cơ 3 pha?
	
	\item Vận hành động cơ 3 pha rotor dây quấn, vẽ sơ đồ mạch nguyên lý?
	
	\item Hoạt động mạch điều khiển động cơ rotor kiểu lấy công suất trượt trả về nguồn?

	\item Cơ sở toán học và nguyên lý thay đổi tốc động cơ không đồng bộ bằng biến tần?
	
	\item Có thể dùng biến tần thay đổi tốc độ động cơ 2 pha hay không, nếu được thì vẽ sơ đồ đấu nối?
	
	\item Tại sao nói phương pháp thay đổi tốc độ động cơ 3 pha không đồng bộ bằng điện áp chỉ có hiệu quả đối với các dạng bơm, quạt,\ldots?
	
	\item Dùng biến tần có phải tiết kiệm điện năng không, tại sao? Cho thí dụ.
	
	\item Có thể dùng biến tần cài đặt tần số thấp cho chạy đồng cơ rotor lồng sóc bình thường thay cho động cơ có hộp giảm tốc không, tại sao?
	
	\item Ưu, nhược điểm của động cơ DC so với động cơ AC rotor lồng sóc?
	
	\item Điện áp ra của biến tần có phải áp $\sin$ không?

	\item Có thể dùng biến tần chuyển từ 1 pha ra 3 pha để làm nguồn 3 pha cấp cho nhà xưởng được hay không, tại sao?
	
	\item Biến tần với Input 220V có cấp nguồn đủ cho động cơ 380V hay không, tại sao?

	\item \ldots\ldots\ldots
\end{enumerate}
\end{document}